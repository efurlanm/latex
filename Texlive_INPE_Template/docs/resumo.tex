%%%%%%%%%%%%%%%%%%%%%%%%%%%%%%%%%%%%%%%%%%%%%%%%%%%%%%%%%%%%%%%%%%%%%%%%%%%%%%%%
% RESUMO %% obrigat�rio

\begin{resumo}

%% neste arquivo resumo.tex
%% o texto do resumo e as palavras-chave t�m que ser em Portugu�s para os documentos escritos em Portugu�s
%% o texto do resumo e as palavras-chave t�m que ser em Ingl�s para os documentos escritos em Ingl�s
%% os nomes dos comandos \begin{resumo}, \end{resumo}, \palavraschave e \palavrachave n�o devem ser alterados

\hypertarget{estilo:resumo}{} %% uso para este Guia

Neste trabalho � analisada a poss�vel natureza ca�tica da turbul�ncia atmosf�rica. As an�lises aqui realizadas, baseadas em dados de temperatura de alta resolu��o, obtidos pela campanha WETAMC do projeto LBA, sugerem a exist�ncia de um comportamento ca�tico de baixa dimens�o na camada limite atmosf�rica. O atrator ca�tico correspondente possui uma dimens�o de correla��o de $D_{2}=3.50\pm0.05$. A presen�a de din�mica ca�tica nos dados analisados � confirmada com a estimativa de um expoente de Lyapunov pequeno mas positivo, com valor $\lambda_{1}=0.050\pm0.002$. No entanto, esta din�mica ca�tica de baixa dimens�o est� associada � presen�a das estruturas coerentes na camada limite atmosf�rica e n�o � turbul�ncia atmosf�rica. Esta afirma��o � evidenciada pelo processo de filtragem por wavelets utilizado nos dados experimentais estudados, que permite separar a contribui��o da estruturas coerentes do sinal turbulento de fundo.

\palavraschave{%
	\palavrachave{Turbul�ncia atmosf�rica}%
	\palavrachave{Campanha WETAMC}%
	\palavrachave{Projeto LBA}%
	\palavrachave{Comportamento ca�tico}%
	\palavrachave{Atrator ca�tico}%
}
 
\end{resumo}